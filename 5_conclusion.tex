\section{Conclusion}\label{conclusion}

%This paper will present the design of an evaluation framework and its application to ADS-B data. 
%The framework is intended to provide researchers and practitioners with practical tools to select appropriate cleaning strategies, balance data fidelity with usability, and build more reliable foundations for trajectory prediction, safety assessment, and air traffic management. 
%In addition, the paper will contribute:
%\begin{itemize}
%  \item an overview of the current use of ADS-B data in the research domain, identifying and structuring clusters of algorithms in use;
%  \item an analysis of the main trends in ADS-B data usage based on a systematic literature review;
%  \item extraction of a representative set of algorithms that rely on ADS-B data and testing their robustness under varying data correction strategies; and
%  \item conclusions on algorithm sensitivity to data cleaning, with a discussion of trade-offs between preprocessing, fidelity to original data, and downstream accuracy, as well as future research directions.
%\end{itemize}

In this paper, we reviewed key application domains of ADS-B data, from trajectory modeling and prediction to methodology, simulation and policy. We outlined the major cleaning procedures, including outlier removal, interpolation, resampling, and smoothing. Using an autoencoder-based case study, we quantitatively assessed how different noise types (Gaussian, drift, and spikes) affect trajectory reconstruction, which indicates, for AE-based reconstruction models, noise suppression in the data cleaning stage should prioritize smoothing or interpolation optimization rather than excessive removal of local outliers, in order to preserve the overall trajectory structure. Future research will further extend the evaluation to multiple algorithm types and performance metrics, aiming to establish a more systematic framework for analyzing the impact of data quality on downstream result.