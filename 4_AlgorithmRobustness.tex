\section{use case: investigation of the noise impact on Auto Encoder algorithm}\label{usecase}

\subsection{Objective}
This section presents a complete case study that constructs artificial noise experiments to quantitatively evaluate the effects of three typical noise types-Gaussian noise, drift, and spikes—on the trajectory reconstruction performance of an auto-encoder (AE) model. The results further help to identify key priorities and directions for the data cleaning stage.

The experiment employs an ADS-B dataset collected at Zurich Airport, covering trajectory data recorded from 04:57:13 (UTC) on October 1, 2019 to 18:57:37 (UTC) on November 30, 2019, as shown in Figure \ref{fig:totaltracks}. The dataset contains approximately 2.8 million ADS-B messages, representing the complete trajectories of about 14,000 flights. Each record includes fields such as timestamp, altitude, longitude, latitude, ground speed, heading, callsign, and ICAO24 code, with longitude ranging from 7.5702 to 9.5276 and latitude from 46.8019 to 48.1302.

% TODO: \usepackage{graphicx} required
\begin{figure}[h]
	\centering
	\includegraphics[width=0.5\linewidth]{totaltracks}
	\caption{Visualization of the whole dataset}
	\label{fig:totaltracks}
\end{figure}

For model training, the data were preprocessed as follows: trajectories were resampled so that each one contained 100 coordinate points, ensuring a uniform input dimension; and normalization was applied to preserve the spatial proportion of coordinates.

\subsection{Auto-encoder Model}
In terms of algorithm selection, this study builds a simple yet stable AE model. The AE learns the latent space representation of flight trajectories by performing unsupervised feature compression and reconstruction, enabling it to reproduce the input trajectory data.

The autoencoder consists of two symmetric subnetworks: an encoder and a decoder. The encoder takes a 200-dimensional preprocessed trajectory vector as input and compresses it through three layers with 128, 64, and 32 neurons, respectively. The last layer outputs a 32-dimensional latent representation, which captures the embedded features of the trajectory in a lower-dimensional space. The decoder mirrors the encoder’s structure, expanding through layers of 32, 64, and 128 neurons, and finally reconstructing the trajectory back to a 200-dimensional output vector.

The model’s reconstruction error is optimized using the Mean Squared Error (MSE) loss function, defined as:

\[
\mathcal{L}_{\text{rec}} = \frac{1}{N} \sum_{i=1}^{N} \| x_i - \hat{x}_i \|^2
\]

, where \( x_i \) denotes the input trajectory vector, \( \hat{x}_i \) is the reconstructed output of the model, and \( N \) is the number of samples.
This loss function reflects the \textit{information fidelity} of the reconstruction process. When the model is well-trained and effectively learns the trajectory behavior features, the reconstructed output $\hat{x}$ closely follows the overall trend of $x$, with only minor deviations in local details.

The dataset is split into training and validation sets in an 8:2 ratio, and the AE is trained in an unsupervised manner under the same normalization conditions. During training, both training and validation losses are continuously monitored to evaluate convergence and prevent overfitting. After approximately 200 epochs, the losses stabilize and converge to $\mathrm{MSE}$ $\approx$ 0.002, indicating that the model successfully learns the main spatial structure of flight trajectories and demonstrates strong reconstruction capability.

\subsection{Baseline Generation}
In this process, a trained AE model is used to identify the trajectories most similar to clean data, which are then selected as the baseline for subsequent experiments. Specifically, the model reconstructs all normalized trajectory samples and computes the reconstruction error for each trajectory. The most representative subset of trajectories with the lowest reconstruction errors is selected as the baseline dataset.

This means that if a trajectory can be reconstructed by the model with an extremely low error, its morphological features are highly representative and regular within the training data distribution, reflecting the typical trajectory patterns learned by the model. Such trajectories are considered the most “reconstructable” samples in the dataset—that is, the ones closest to clean data.

As shown in Figure \ref{fig:best100}, the baseline trajectories exhibit high smoothness and spatial consistency, conforming to the physical laws of real flight paths. In contrast, high-error samples, showed in Figure \ref{fig:worst100}, often contain data anomalies or noisy points. Therefore, by leveraging the AE model’s reconstruction error as a selection criterion, it becomes possible to automatically identify high-quality trajectory data without manual thresholds or interpolation procedures. This method helps avoid errors or inappropriate parameter settings that can arise during manual preprocessing, thereby producing a statistically sound and model-adaptive baseline dataset. In total, 100 high-quality trajectories were selected as the baseline dataset for the experiment.

\begin{figure}
	\centering
	\begin{subfigure}[t]{0.4\textwidth}
		\centering
		\includegraphics[width=\textwidth]{best100}
		\caption{The best 100 trajectories}  
		\label{fig:best100}  
	\end{subfigure}
\hspace{0.05\textwidth}	
\begin{subfigure}[t]{0.45\textwidth}
	\centering
	\includegraphics[width=\textwidth]{worst100}
	\caption{The worst 100 trajectories}  
	\label{fig:worst100}
\end{subfigure}
\caption{Visualization of reconstructed trajectories}
\label{conparison of best and worst}
\end{figure}

\subsection{Noise Injection}
To systematically analyze the impact of different noise types on AE reconstruction performance, three typical artificial noise sources were injected into the baseline trajectories: Gaussian noise, drift noise, and spike noise. These correspond to the most common sources of error in ADS-B data and collectively simulate disturbances that occur during data acquisition, transmission, and decoding. The experiment procedure is shown in Figure \ref{fig:experiment-structure}.
\begin{figure}
	\centering
	\includegraphics[width=0.6\linewidth]{"experiment structure"}
	\caption{experiment procedure}
	\label{fig:experiment-structure}
\end{figure}

Gaussian noise simulates random measurement errors affecting ADS-B signals during reception or localization—such as GNSS range errors, multipath propagation, or receiver thermal noise—resulting in small position fluctuations. It is implemented by adding Gaussian perturbations to each coordinate dimension of the baseline trajectories.

Drift noise represents time-dependent cumulative deviations caused by positioning or clock synchronization errors, commonly manifested as gradual longitude or latitude shifts over time. This type of error may originate from GNSS reference drift, sensor calibration bias, or timestamp misalignment. It is simulated by superimposing a small linear offset proportional to time on the trajectory coordinates.

Spike noise corresponds to sporadic outliers or sudden jumps in ADS-B messages, such as decoding errors, packet loss, or transient interference leading to abrupt changes in altitude or speed. In the experiment, random subsets of points were selected from each trajectory and perturbed with sudden amplitude changes of random magnitude.

Each of the three noise types was configured with multiple intensity levels to observe how varying magnitudes of error affect model performance. As illustrated in Figure \ref{fig:noisy-tracks}, Gaussian noise causes overall jitter while preserving trajectory shape; drift noise induces gradual spatial displacement over time; and spike noise introduces localized abrupt deviations.
\begin{figure}
	\centering
	\includegraphics[width=0.5\linewidth]{"noisy tracks"}
	\caption{Trajectories with added noise}
	\label{fig:noisy-tracks}
\end{figure}

The resulting noisy trajectories thus possess controlled and interpretable noise characteristics, closely corresponding to realistic ADS-B error patterns and providing a solid foundation for robustness analysis of the AE model.

\subsection{Metrics and Results}
To quantitatively evaluate the AE model’s reconstruction robustness, the Root Mean Squared Error (RMSE) was adopted as the primary evaluation metric, measuring the overall deviation between the reconstructed and the true noisy trajectories:

\[
\mathrm{RMSE} = \sqrt{\frac{1}{N} \sum_{i=1}^{N} \left( x_i - \hat{x}_i \right)^2}
\]

where \( x_i \) is the model input, \( \hat{x}_i \) is the reconstructed output, and \( N \) is the number of samples. A smaller RMSE indicates that the model's reconstruction is closer to the true trajectory, implying stronger robustness.

For each noise type, experiments were conducted under multiple noise intensity parameters, and the RMSE distribution for each condition was recorded. All tests were performed using the same AE model configuration to ensure that performance variations stem solely from noise perturbations. To enhance statistical reliability, each experiment was repeated 10 times, and the results were averaged.

As shown in Figure \ref{fig:impactofnoise}, the results indicate that:

In the Gaussian noise group, increasing the noise level ($\sigma$) leads to a clear rise in RMSE and greater variance, suggesting high sensitivity to random high-frequency perturbations.

In the drift noise group, RMSE increases slightly with stronger drift, but the overall growth remains slow, indicating that the AE model can capture the global trajectory trend and tolerate gradual cumulative deviations.

In the spike noise group, RMSE shows the smallest fluctuations and remains nearly stable, rising only slightly when the spike probability is high, implying that occasional outliers have limited effect on overall reconstruction.

In summary, the AE model is most sensitive to Gaussian-type random noise, while showing greater tolerance to drift and spike noise. This suggests that, during the ADS-B data cleaning stage, attention should focus on smoothing and reducing high-frequency noise (e.g., localization jitter or measurement fluctuation). In contrast, short-term anomalies or mild drifts can often be mitigated by the model itself. Therefore, noise suppression strategies should emphasize smoothing filters and interpolation optimization, rather than excessive removal of local outliers, in order to preserve the structural integrity of trajectories.

\begin{figure}
	\centering
	\includegraphics[width=1\linewidth]{impact_of_noise}
	\caption{boxline of result}
	\label{fig:impactofnoise}
\end{figure}

\subsection{Discussion}
The choice of the AE as the analytical model was mainly motivated by its simple architecture, unsupervised learning capability, and strong performance in trajectory reconstruction tasks. However, it is important to note that the AE serves only as an illustrative example to reveal the relationship between noise and algorithmic performance. The impact of noise strongly depends on the model’s structure and learning mechanism—models such as LSTM, CNN, or Kalman filters may demonstrate markedly different levels of robustness. Future research will therefore extend the comparative analysis to various model architectures to derive more generalizable conclusions.

Similarly, regarding evaluation metrics, while RMSE effectively measures reconstruction accuracy, additional indicators such as temporal consistency, structural similarity, and feature preservation rate may be introduced in future work to better assess performance in trajectory prediction, classification, or anomaly detection tasks.

Moreover, since the AE model requires a fixed input dimension, this study did not further investigate the effect of missing-value errors on reconstruction performance, which is typically addressed by interpolation or resampling techniques during data cleaning. Future studies can incorporate different data imputation strategies within this framework for comparative analysis.

In conclusion, the experiments and analyses in this chapter validate the AE model’s reconstruction behavior under various noise conditions, providing empirical evidence for understanding how data quality influences algorithmic performance.
Future research will build upon this work by expanding the range of model types, evaluation metrics, and noise modeling complexity, ultimately establishing a comprehensive and interpretable evaluation framework for assessing the impact of ADS-B data cleaning on downstream algorithmic performance.