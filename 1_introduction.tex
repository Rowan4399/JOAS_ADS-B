\section{Introduction}

ADS-B has become a crucial data source for air traffic management, supporting a wide range of applications. Typical studies include trajectory prediction~\cite{wang2021performance,corrado2021clustering}, flight phase identification~\cite{schlosser2024analysis}, trajectory clustering and modeling~\cite{guleriamachine}, safety analysis~\cite{noh2018aviation} (e.g., conflict detection and collision risk assessment), and airport operations optimization~\cite{roosenbrand2023contrail} (e.g., runway occupancy and taxiing analysis). ADS-B has also become a key enabler for environmental studies such as fuel consumption estimation, contrail detection, and large-scale emissions assessment.
Despite these advances, ADS-B data quality remains a significant concern. Missing points, irregular sampling, and anomalies complicate processing and may bias analysis. To improve usability, researchers apply cleaning methods such as interpolation, smoothing, outlier removal, and resampling. However, these methods can also distort the statistical and physical characteristics of trajectories. For example, interpolation can mask subtle variations and increase prediction errors, while outlier removal may discard rare but genuine safety-critical events. Yet, existing studies rarely provide systematic and quantitative analysis of how different strategies influence downstream results.
To address this gap, this study systematically evaluates the relationship between data cleaning and algorithmic performance in ADS-B analytics. Chapter \ref{soa} reviews eight major application domains of ADS-B data to contextualize its analytical value. Chapter \ref{clean}  summarizes common data cleaning techniques and proposes a generalized preprocessing pipeline integrating detection, interpolation, and smoothing. Chapter \ref{usecase} presents an autoencoder-based case study that quantifies the impact of different noise types (Gaussian, drift, and spikes) on trajectory reconstruction performance, followed by a discussion of the observed impacts and implications. Finally, Chapter \ref{conclusion} concludes the study and outlines directions for future research. These analyses aim to provide a clearer understanding of how data quality shapes learning-based ADS-B algorithms and to inform the design of more robust data processing strategies.