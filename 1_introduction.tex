\section{Introduction}

With the rapid growth of global air traffic, traditional radar surveillance systems, such as primary surveillance radar (PSR) and secondary surveillance radar (SSR), have shown increasing limitations in coverage, accuracy, and cost. To address these issues, ADS-B technology was developed. Relying on satellite navigation and onboard sensors, ADS-B uses the Global Navigation Satellite System (GNSS) to obtain position and velocity information, which is then integrated with barometric altitude, inertial navigation, and airspeed measurements to generate flight state data. This information is periodically broadcast via ADS-B devices, including identifiers, position, altitude, velocity, and flight intent~\cite{olive2024filtering}. Compared with traditional radar, ADS-B significantly enhances situational awareness for both pilots and air traffic controllers, while reducing infrastructure and maintenance costs.

The development of ADS-B can be traced back to the 1970s, with its concept first proposed and later validated in the U.S. FAA’s Safe Flight 21 project during the 1990s. The “Capstone Program” in Alaska (1999–2006)~\cite{faa2000capstone} further demonstrated its ability to improve safety and efficiency in remote airspace. In 2003, the 11th ICAO Air Navigation Conference officially recognized ADS-B as a surveillance method and promoted standardization. Since the 2010s, ADS-B has entered large-scale deployment: the U.S. FAA mandated “ADS-B Out”~\cite{cfr91-225}, Europe implemented ADS-B under the SESAR framework~\cite{undertaking2009european}, and Australia, Singapore, and other countries followed suit, with China issuing its national implementation plan in 2015~\cite{caac2015adsb}. More recently, space-based ADS-B technology~\cite{melero2024satera} has extended global coverage, while open platforms such as OpenSky Network~\cite{schafer2014bringing} have improved data accessibility and research value.

Beyond surveillance, ADS-B data are widely used in research and operational contexts. Typical studies include trajectory prediction~\cite{wang2021performance,corrado2021clustering}, flight phase identification~\cite{schlosser2024analysis}, trajectory clustering and modeling~\cite{guleriamachine}, safety analysis~\cite{noh2018aviation} (e.g., conflict detection and collision risk assessment), and airport operations optimization~\cite{roosenbrand2023contrail} (e.g., runway occupancy and taxiing analysis). ADS-B has also become a key enabler for environmental studies such as fuel consumption estimation, contrail detection, and large-scale emissions assessment.

Despite these advances, ADS-B data quality remains a significant concern. Missing points, irregular sampling, and anomalies complicate processing and may bias analysis. To improve usability, researchers apply cleaning methods such as interpolation, smoothing, outlier removal, and resampling. However, these methods can also distort the statistical and physical characteristics of trajectories. For example, interpolation can mask subtle variations and increase prediction errors, while outlier removal may discard rare but genuine safety-critical events. These examples illustrate the conditional and sometimes counterproductive impact of cleaning on algorithm performance. Yet, existing studies rarely provide systematic and quantitative analysis of how different strategies influence downstream results.

To address this gap, we propose a structured, indicator-driven evaluation framework. It integrates:

\begin{itemize}
  \item a multi-dimensional metric system to quantify data quality (e.g., completeness, reliability, consistency);
  \item controlled datasets of varying quality, generated from reference trajectories with injected anomalies; and
  \item comparative experiments under a unified algorithmic environment (e.g., trajectory prediction and air traffic management models).
\end{itemize}

This design enables robust, quantitative assessment of how data cleaning strategies affect algorithm accuracy, reliability, and reproducibility.